\documentclass[11pt]{article}
\usepackage{graphicx}
\usepackage{hyperref}
\renewcommand{\refname}{Literature References}
\begin{document}
 \begin{center}
\begin{figure}[h!]

\centering \huge \textbf{MAKERERE\includegraphics[scale=0.1]{image}UNIVERSITY}
\end{figure}
 \begin{large}
{COLLEGE OF COMPUTING  INFORMATION SCIENCE}\\
BIT 2207 RESEARCH METHODOLOGY\\


MUZEEYI JOHN \\16/U/7882/PS 216005151\\
Assignment4: Literature Review

\end{large}
\end{center}
\newpage
\begin{huge}
 Literature Review: GOOGLE MAPS
\end{huge}
\section{Introduction:}
In today's life, Time and Money are very important; no one wants to waste their time and money.
 Google maps has been a great application to the travellers. The user can find the shorter routes. Google maps help users to choose different types of transportation i.e. bus, car, etc. 

\section{overview}
Google maps is a web based navigation system developed by Google. It allows the users to search for different places around the world.\cite{Wiki2016} \\
  Google maps is used for getting locations of different cities.  Google maps gives different options to the user to select their mode of transportation i.e. Bus,Train and walking.   
 Google maps also gives the distance and time for traveling one place to another to the user. \\
 Google maps helps the user by providing directions during driving, public transportation and walking directions. It also provides with the Live traffic conditions, incident reports, and automatic rerouting to find the best route. It also provides Street View and Satellite view.\cite{Suvarna2016}  
 \paragraph{The major disadvantage of Google maps is that when the user wants to travel multiple destinations then Google maps provides directions as per the locations entered by the user but it does not provide the optimized route for these multiple destinations.\cite{Suvarna2016}}
 \section{Conclusion}
 There is need for multiple destination planning feature on Google maps, so as to increase time and money saving.
 
  \begin{thebibliography}{9}
   \bibitem{Wiki2016}Wiki Pedia. \textit{"Google Maps "},March, 2016.\\
 Internet:$https://www.en.wikipedia.org/wiki/Google_Maps$.
    \bibitem{Suvarna2016} Prof. Suvarna Pansambal, Nikhil Iyer, Vira Meherkar, Pankaj Sharma ,International Journal of Advanced Research in Computer and Communication Engineering. \textit{Navigation Systems  for Fastest Route.} vol 5,issue3,March,2016,
    Internet:https://www.ijarcce.com/upload/2016/march-16/IJARCCE%20120.pdf$.
 



\end{thebibliography}
\end{document}